Driving Style Recognition and Driver Behavior Recognition had well study field in intelligent Driver Assistance Systems (DAS) and IEEE classification of Intelligent Transportation Systems. The MIROAD  \cite{johnson2011driving} system discussed was most cited paper which used of smartphone as sensor platform to capture a driving behaviour and fusing related inter-axial data from multiple sensors into a single classifier based on the Dynamic Time Warping (DTW) algorithm. Moreover, the idea of capturing driving data and using Machine Learning techniques such as Graphical models, HMMs and potentially extensions (CHMMs) had discussion in \cite{oliver2000graphical}. In fact, smartphone-based sensing in vehicles for intelligent transportation system applications had well discussion in \cite{fill1, fill2, fill3, fill4, fill5, fill6} and many Machine Learning algorithms had explored. 
\\
However, our approach aimed at encouragement of a driver (a user) participation. One of the design goal for the SEDA was self-situation-awareness rather than driving monitoring device for enforcing use as discussed in these papers. Furthermore, we also observed that a smartphone as a sensor platform was not power efficient. Intuitively, a smartphone were not designed for a sensor platform use. Therefore, the SEDA had put in this consideration where we decomposed the SEDA as a main sensing device, worked in conjunction with a user smartphone and a cloud computing for intelligent processing.
\\

%%In the following sections we explored the details of the SEDA in Hardware and Software aspects.