% CHI Extended Abstracts template.
% Tested with XeTeX on Mac OS X (Get it from http://tug.org/mactex)
% The latest version is available at <http://manas.tungare.name/software/latex/>
% 
% Filename: chi-ext.cls
% 
% CHANGELOG:
%   2010-10-18   Manas Tungare      Restored support for \figures.
%   2010-08-09   Manas Tungare      Updated copyright info for CHI 2011
%   2009-12-04   Stephen Voida      Updated copyright info for CHI 2010
%   2008-11-25   Manas Tungare      Initial create.
%   2009-11-17   Manas Tungare      Refactored the title & author sections.
% 
% LICENSE:
%   Public domain: You are free to do whatever you want with this template.
%   If you improve this in any way, please drop me a note <manas@tungare.name>,
%   so I can share the updates with everyone.
%   
%   PLEASE RECONSIDER BEFORE FORKING THIS TEMPLATE; there are already
%   several versions of the chiproceedings template for no good reason.
%   DO NOT REDISTRIBUTE THIS FILE UNDER A DIFFERENT FILENAME unless you
%   have a very good reason to change its name.

\documentclass{chi-ext}

\usepackage{subfiles}
\usepackage{wrapfig}
\usepackage[export]{adjustbox}
\usepackage{graphicx}
\graphicspath{ {images/} }


\title{S.E.D.A - Smart Elegant Driving Assistant}

\author{
  \textbf{San Kho Lin} \\
  829463 \\
  sanl1@student.unimelb.edu.au \\
  \\
  \textbf{Bingfeng Liu} \\
  639187 \\
  bingfengl@student.unimelb.edu.au \\
  \\
  \textbf{Yixin Chen} \\
  522819 \\
  alachen@student.unimelb.edu.au \\
}

\keywords{active decision making, safer future, safe driving, driver assistance systems, intelligent transportation systems, augmented reality, machine learning, artificial intelligence}

%\acmclassification{H.5.2 Information Interfaces and Presentation: User interfaces –
%Evaluation/ methodology}

\copyrightinfo{
  Copyright is held by the author/owner(s). \\
  \emph{SEDA Team}, Mobile Computing Systems Programming. \\
  The University of Melbourne. COMP90018 2017SM2 \\
}

% Repeat author names (minus affiliations and addresses) and title here 
% for PDF metadata.
\hypersetup{
  pdfauthor={San Kho Lin, Bingfeng Liu, Yixin Chen},
  %pdfkeywords={Keyword 1, Keyword 2},
  %pdfsubject={General Subject Area},
  pdftitle={S.E.D.A - Smart Elegant Driving Assistant},
}

\begin{document}
\maketitle

\begin{multicols}{2}
  
\makeauthors
\makecopyright

\section{Abstract}
\subfile{abstract}

\section{Keywords}
\makeatletter \@keywords \makeatother

%\section{ACM Classification Keywords}
%\makeatletter \@acmclassification \makeatother

%------------------------------------------------------------------------

\section{Introduction}
\subfile{intro}

\section{Background}
\subfile{background}

\section{Related Work}
\subfile{related-work}

\section{Hardware Aspect}
\subfile{hardware}

\section{Software Aspect}
\subfile{software}

\section{Extension of the System}
\subfile{extension}

\section{User of the System}
The major design goal of the SEDA encouraged the participation of the user. Therefore, the SEDA was designed solely for using as a personal gadget that everyone could buy at local consumer electronic store. A user with enough technical background could able to operate their own instance of the SEDA cloud intelligence service by following the operational manual. Otherwise, a user subscribed to the third party service provider for secured access of their private driving profile data. This idea reassembled to the popular sport wear Fitbit sharing of a user sport activities and their health statistics.

Besides as a personal gadget, if the user decided to disclose their driving profile (a good driving behaviour) for a car insurance company, this could negotiate an incentive for insurance premium as a proven evidence for safe and good behaving driving record.

The enforcing body like VicRoads could also use the SEDA to train a better safer driver.

Furthermore, the SEDA could more generalized its use by not only for a car driver but also for any driver in a good behaving driving needs such as Cyclist, Motorbike and so on.

\section{Conclusion}
\subfile{conclusion}

\bibliography{papers}
\bibliographystyle{abbrv}

\end{multicols}

\newpage
\section{Appendix}
\subfile{appendix}


\end{document}