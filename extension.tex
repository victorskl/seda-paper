
\documentclass[main.tex]{subfiles}
\begin{document}

%\subsection{Extension}

% *** Ideas of extension ***
% - Distributed Systems
% - Gaze
% - Smart watch
% - Telematic Devices (such as monitoring Tyre worn off)

\subsection{Car’s internal state}

The SEDA was not able to detect the internal state of the car (e.g. Gear and steering wheel angle). The sensors could be embedded into the cars and monitoring the car’s status in real time with the internet of things technologies. Those data could be used by the SEDA system to improve the safe driving advice to the driver. The internal car data could detect the potential failure of the car component and warned the driver to drive safely with respect to those failures, for example, if the sensor detected the car brake was malfunction the system could ask the driver to drive slowly or pull over \cite{oliver2000graphical}.

By extending the monitoring of the internal car state, the idea of the smart car could be developed that our device could pair to control the car if there were any extremely abnormal activities of the driver detected by the SEDA. The study \cite{johnson2011driving} mentioned the idea of categorizing driver into nonaggressive and aggressive via analyzing the driving events. Some drivers did not realize that they were committing potentially-aggressive actions. The aggressive actions included excessive speeding, improper following, erratic lane change, etc. If The driver kept violating the safe driving advice given by the SEDA AI and persistently performing dangerous driving actions, The SEDA AI could override the car internal state such that slow down the car or limit the maximum speed.

\subsection{SEDA + Smart Watch}

A wearable device like Smart watch had the advantage of closely monitoring user’s body activity for a long time \cite{johnson2014literature}. Smart phone also had higher chance to stay with the user, but at most of the time its capability of monitoring user’s motion was limited since smart phone were mostly staying in our pocket or bag. Since, wearable device like smart watch could be worn at user’s arm thus smart watch could detect richer body motions. The SDEA device needed to stay on user’s head; it couldn’t be too heavy otherwise the driver will feel uncomfortable while driving. The smart watch could install more sensors, powerful CPU and large battery, since arms tolerated with slightly heavier objects \cite{johnson2014literature}. The smart watch was a suitable extension for our SEDA advisory system, for example detecting drowsiness of the driver. As we could imagine it would be hard to only use head movement alone to detect the sleepy driver since the input factor of head movement was too simple in this case.  In \cite{rios2015variation}, the authors were proposing a novel method of detecting drowsiness rather than the heavy EEG-based (electroencephalographic) system. By using the pedometer, accelerometer, gyroscope and heartbeat sensors from the commercial smart watch, the SDEA system could detect drowsiness of the driver and trigger the vibration sensors in the smart watch to wake the driver and warn him/her the dangerous of fatigued driving. There were about 10-30\% of traffic accidents directly related to sleepiness and fatigue while driving, so detecting drowsiness of driver was important to increase the road safety.

\end{document}